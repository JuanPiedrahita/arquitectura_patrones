\chapter{Trabajos Futuros}

La siguiente versión de este proyecto estará enfocada a convertir a JSJSports web en una plataforma e-commerce general que no solo se aplique a la venta de artículos deportivos sino que en esta se pueda ofertar cualquier tipo de artículo y se cuente con un vendedor que brinde soporte al cliente durante el proceso de compra, para esto es necesario aumentar el número de módulos y de productos disponibles, es decir, diversificar la página.\newline
\\
Pero esta diversificación no solo se logrará agregando módulos y productos sino que también se hace necesario que se diversifiquen las formas de pago que se tienen actualmente y por qué no se permita una interacción más cercana con el cliente mejorando los canales de comunicación e implementando nuevos métodos como las videollamadas.\\
\\
Al igual que se explica en el punto de vista de Migración de la capa de proyecto analizada en el presente trabajo como arquitectura final  del e-commerce se plantea una plataforma que proporcione un servicio para diferentes páginas, es decir que no sea necesario que el cliente se encuentre en el dominio de la página sino que este pueda consultar productos por medio de otras o de terceros al igual que para que para que el cliente vea la publicidad no necesariamente tenga que estar en el dominio de la página.