\chapter{Conclusiones}


Usar Archimate como lenguaje de modelado para todo el diseño y desarrollo del proyecto permitió comprender y analizar con cada punto de vista todos y cada uno de los aspectos que tiene el proyecto y que dejo como resultado una lista de diagramas que son de fácil lectura y comprensión para cualquier persona que conozca el lenguaje o que tenga solo las nociones principales.\\
Cada una de las cinco capas de Archimate nos permitió modelar el sistema desde diferentes puntos de vista y no solo desde el punto de vista del desarrollador, permitiendo así tener en cuenta aspectos que son escenciales para el negocio y que un desarrollador podría omitir por diferentes causales. \\
Se observo para cada capa lo siguiente:
\begin{itemize}
	\item \textbf{Capa de Negocio o Empresarial: } La capa de negocio permite modelar todo el negocio desde los actores y ubicaciones con las que se cuentan hasta los productos y servicios que se brindan.
	\item \textbf{Capa de Apicación: }La capa de aplicación es compatible con la capa de negocio pero esta vez implementados en el software.
	\item \textbf{Capa de Tecnología o Infraestructura: }La capa de tecnología nos permitió modelar el sistema desde un punto de vista de lo necesario para li implementación del mismo ayudando así a entender todos los aspectos involucrado y necesarios para llevar a cabo este proceso.
	\item \textbf{Capa de Motivación: }La capa de Motivación nos permitió modelar las razones por las que se esta planteando el desarrollo del proyecto.
	\item \textbf{Capa de Proyecto: }La capa de proyecto nos permitió modelar el desarrollo actual del producto y las arquitecturas futuras que se desean como parte todo de un solo proceso que se lleva a cabo en una secuencia. 
\end{itemize}

Durante el desarrollo del análisis y durante la implementación del proyecto se trabajo con un enfoque de componentes, usar este enfoque trae consigo varias ventajas como que no solo se logra que el sistema tenga un grado  bajo de acoplamiento sino que también se permite la extensión de este fácilmente debido a que no solo se ordena el sistema y se hace más fácil de entender sino que también agregar componentes es más fácil que agregar códigos. Ademas de la extensión se tiene que destacar que se facilita el mantenimiento del software.\\
\\
Al usar los patrones de diseño no solo se ve que los diseños y la implementación se hacen más fáciles y entendibles sino que también se observa que a futuro hacer usos de estos permitirá extender las funcionalidades del sistema de una forma mucho más fácil y que no será necesario que la misma persona lo haga debido a que se sigue un estandar que es entendible para todos.